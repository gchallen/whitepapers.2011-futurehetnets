% 1) name and affiliation

Currently I am a post-doctoral scholar supervised by Professor Hari
Balakrishnan at MIT. In August, 2011, I will begin a junior faculty position
at the University at Buffalo.

% 2) the scientific contributions the candidate is likely to make to the
% workshop, including a brief description of areas of expertise

\uline{I am interested in power as a design constraint spanning all classes
of computer systems} --- tiny embedded sensors, mobile phones, personal
computers, datacenters, and large-scale networks. My research agenda is
rooted in the belief that studying power consumption and management at
multiple levels of the computing infrastructure will yield insights relevant
\textit{at all levels}. Future heterogeneous networks will interweave
multiple classes of devices where power is important. At {\scshape
FutureHetNets}, I will contribute my interest and expertise in this area.

My graduate research completed earlier this year focused on energy management
for wireless sensor networks. I built three systems --- Lance, IDEA and
Peloton --- that manage energy at the \textit{network}, rather than node,
level. Lance improves the performance of data-intensive sensor network
applications by considering both the cost and value of information when
collecting data. IDEA provides a network-wide energy coordination layer
facilitating energy optimizations impossible for a single node to perform
alone. Peloton proposes a distributed operating system for coordinated
resource management built on state sharing, a distributed energy ticket
abstraction, and local neighborhood ticket management. Along similar lines, I
contributed to the development of PowerTOSSIM, which added power modeling and
to the TinyOS simulator, and the Pixie sensor node operating system, which
promoted energy to a first-class system resource.

At MIT I am working with Prof. Balakrishnan on techniques to save power in
802.11 networks. At present we are experimenting with physical-layer packet
source detection. If accurate, this could allow clients to detect and ignore
packets that are not addressed to them, avoiding the energy overhead of full
packet decoding and allowing the radio to remain in a low-power state. With a
collaborator at Drexel University I am also addressing the complexities of
operating the next generation of heterogeneous low-power hardware
architectures, which will incorporate multiple components with different
power-performance tradeoffs.

% 3) reasons why participation in the workshop will benefit the
% candidate

At this early stage in my professional career interaction with other
scientists is extremely fruitful. While I am interested in energy management,
I am still developing this focus into a coherent research agenda.
Participation in {\scshape FutureHetNets} will provide valuable exposure to
experts in this area. I will attend as an enthusiastic and informed
participant, and have a history of vocal participation in similar venues,
such as HotOS.

% 4) a statement indicating whether NSF travel support is
% requested or is not requested.

Because my appointment at Buffalo has not yet began and I do not have funding
available this spring for research travel, I would appreciate travel support
to attend {\scshape FutureHetNets}. I will happily volunteer my time in some
useful capacity during the workshop.
